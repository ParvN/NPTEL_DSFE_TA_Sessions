\documentclass[11pt,a4paper]{article}
\usepackage[utf8]{inputenc}
\usepackage{amsmath}
\usepackage{amsfonts}
\usepackage{amssymb}
\usepackage{amsthm}
\usepackage{geometry}
\usepackage{fancyhdr}
\usepackage{graphicx}
\usepackage{enumitem}
\usepackage{xcolor}
\usepackage{tcolorbox}
\usepackage{listings}
\usepackage{hyperref}

\geometry{margin=1in}
\pagestyle{fancy}
\fancyhf{}
\rhead{NPTEL PMRF Live Sessions}
\lhead{DataScience For Engineers}
\cfoot{\thepage}

\newtcolorbox{defbox}[1][]{
    colback=gray!5!white,
    colframe=gray!75!black,
    title=#1,
    fonttitle=\bfseries
}

\newtcolorbox{quesbox}[1][]{
    colback=lightgray!5!white,
    colframe=lightgray!75!black,
    title=#1,
    fonttitle=\bfseries
}

\newtcolorbox{examplebox}[1][]{
    colback=blue!5!white,
    colframe=blue!75!black,
    title=#1,
    fonttitle=\bfseries
}

\theoremstyle{definition}
\newtheorem{definition}{Definition}[section]
\newtheorem{example}{Example}[section]

\title{\color{blue}Data Science For Engineers\\NPTEL PMRF Live Sessions\\}
\author{Teaching Assistant: Parvathy Neelakandan\\PhD Student}
\date{26-08-2025}

\begin{document}

\maketitle

\section*{Multivariate Optimization with Equality Constraints}

\textbf{Why constrained optimization?}
In real-world problems, we need to find the best solution while satisfying certain conditions mostly. 
For example, if you want to optimize your expenses, you might set a constraint that the total must not exceed 10,000.

\begin{defbox}[Constrained Optimization Problem]
A constrained optimization problem with equality constraints has the form:
\vspace{50mm}
\end{defbox}


\begin{defbox}[Lagrangian Function]
For the constrained optimization problem, the Lagrangian function is defined as:
\vspace{30mm}
\end{defbox}

\begin{defbox}[Optimality Conditions]
At the optimal point,
\vspace{40mm}
\end{defbox}

\begin{examplebox}[Practice questions]
Find the maximum and minimum of $ f(x,y) = 36x^2 + y^2-9$ subject to the constraint $3x^2 + y^2 =4$.
\vspace{50mm}
\end{examplebox}

\subsubsection*{Geometric Interpretation}
\noindent \textbf{What does this mean geometrically?}
At the optimal point, the objective function and constraint curves are tangent to each other. 
\begin{quesbox}[Question]

The geometric interpretation of the Lagrange multiplier method is that at the optimum:

(a) The constraint curve is perpendicular to the objective function

(b) The gradient of the objective function is parallel to the gradient of the constraint

(c) The objective function equals the constraint function

(d) Both gradients are zero
\end{quesbox}


\subsection*{Multiple Equality Constraints}

For problems with multiple equality constraints,

\begin{defbox}[Multiple Constraints]
\vspace{30mm}
\end{defbox}

\begin{examplebox}[Practice question]
Find the maximum and minimum of $ f(x,y,z) = 3x^2 + y $ subject to the constraint $4x-3y=9, \text{and} x^2+z^2=9 $.
\vspace{50mm}
\end{examplebox}


\section*{Multivariate Optimization with Inequality Constraints}

These problems are more complex because we must determine which constraints are active or inactive at the optimal solution.

\begin{defbox}[Constrained Optimization Problem]
\vspace{40mm}
\end{defbox}

\begin{defbox}[Karush-Kuhn-Tucker (KKT) Conditions]

\vspace{90mm}
\end{defbox}

\begin{defbox}[Active and inactive constraints]
\begin{itemize}
    \item \textbf{Active constraint}: 
    \vspace{10mm}
    \item \textbf{Inactive constraint}: 
    \vspace{10mm}
\end{itemize}
\end{defbox}

\begin{examplebox}[Practice question]
Find the maximum and minimum of $ f(x,y) = x^2 + y^2 - 4x -6y + 10 $ subject to the constraint $x+y\leq 3 $.
\vspace{50mm}
\end{examplebox}

\begin{quesbox}[Question]
For a problem with $n$ variables and $m$ equality constraints, the Lagrangian method produces a system of:

(a) $n$ equations in $n$ unknowns

(b) $m$ equations in $m$ unknowns

(c) $(n+m)$ equations in $(n+m)$ unknowns

(d) $n \times m$ equations
\end{quesbox}


\section*{Introduction to Data Science}

\begin{defbox}
\textbf{1. Classification Problems:} Assign labels 
\begin{itemize}
\item Binary classification: Two categories (e.g., hypertension/not)
\item Multiclass classification: Multiple categories (e.g., image recognition with many animals (cat, dog, etc.))
\end{itemize}
\textbf{2. Function Approximation Problems:} Find mathematical relationships between input variables and output responses.
\end{defbox}

\subsection*{Why are there so many models?}
\vspace{50mm}

\subsection*{Data Analysis Problem-Solving Framework}

\begin{defbox}[Five-Step Problem-Solving Framework]
\begin{enumerate}
    \item \textbf{Problem Definition}
    \item \textbf{Problem Characterization}
    \item \textbf{Solution Conceptualization}
    \item \textbf{Method Identification}
    \item \textbf{Implementation \& Assessment}
\end{enumerate}
\end{defbox}


\subsection*{Data Imputation}

Data Imputation is the process of filling in missing values in datasets.


\textbf{Matrix Methods for Data Imputation}

Matrix completion techniques can be used to fill missing data by the concepts of null space and pseudo-inverse. 
But then it may not work well in all situations. What alternative methods can be used?


\end{document}