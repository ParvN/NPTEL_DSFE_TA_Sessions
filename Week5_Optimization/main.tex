\documentclass[11pt,a4paper]{article}
\usepackage[utf8]{inputenc}
\usepackage{amsmath}
\usepackage{amsfonts}
\usepackage{amssymb}
\usepackage{amsthm}
\usepackage{geometry}
\usepackage{fancyhdr}
\usepackage{graphicx}
\usepackage{enumitem}
\usepackage{xcolor}
\usepackage{tcolorbox}
\usepackage{listings}
\usepackage{hyperref}

\geometry{margin=1in}
\pagestyle{fancy}
\fancyhf{}
\rhead{NPTEL PMRF Live Sessions}
\lhead{DataScience For Engineers}
\cfoot{\thepage}

\newtcolorbox{defbox}[1][]{
    colback=gray!5!white,
    colframe=gray!75!black,
    title=#1,
    fonttitle=\bfseries
}

\newtcolorbox{quesbox}[1][]{
    colback=lightgray!5!white,
    colframe=lightgray!75!black,
    title=#1,
    fonttitle=\bfseries
}

\newtcolorbox{examplebox}[1][]{
    colback=blue!5!white,
    colframe=blue!75!black,
    title=#1,
    fonttitle=\bfseries
}

\theoremstyle{definition}
\newtheorem{definition}{Definition}[section]
\newtheorem{example}{Example}[section]

\title{\color{blue}Data Science For Engineers\\NPTEL PMRF Live Sessions\\}
\author{Teaching Assistant: Parvathy Neelakandan\\PhD Student}
\date{26-08-2025}

\begin{document}

\maketitle

\section*{Multivariate Optimization with Equality Constraints}

\textbf{Why constrained optimization?}
\vspace{20mm}

\begin{defbox}[Constrained Optimization Problem]
A constrained optimization problem with equality constraints has the form:
\vspace{50mm}
\end{defbox}


\begin{defbox}[Lagrangian Function]
For the constrained optimization problem, the Lagrangian function is defined as:
\vspace{30mm}
\end{defbox}

\begin{defbox}[Optimality Conditions]
At the optimal point,
\vspace{40mm}
\end{defbox}

\begin{examplebox}[Practice questions]
\vspace{160mm}
\end{examplebox}

\subsubsection*{Geometric Interpretation}

The constrained optimum occurs where the gradient of the objective function is a linear combination of the gradients of the constraint functions. 
\vspace{5mm}

\noindent \textbf{What does this mean geometrically?}
\vspace{70mm}
\begin{examplebox}[Practice question]
\vspace{70mm}
\end{examplebox}


\subsection*{Multiple Equality Constraints}

For problems with multiple equality constraints,

\begin{defbox}[Multiple Constraints]
\vspace{30mm}
\end{defbox}

\begin{examplebox}[Practice question]
\vspace{70mm}
\end{examplebox}


\section*{Multivariate Optimization with Inequality Constraints}

These problems are more complex because we must determine which constraints are active or inactive at the optimal solution.

\begin{defbox}[Constrained Optimization Problem]
\vspace{40mm}
\end{defbox}

\begin{defbox}[Karush-Kuhn-Tucker (KKT) Conditions]

\vspace{90mm}
\end{defbox}

\begin{defbox}[Active and inactive constraints]
\begin{itemize}
    \item \textbf{Active constraint}: 
    \vspace{10mm}
    \item \textbf{Inactive constraint}: 
    \vspace{10mm}
\end{itemize}
\end{defbox}

\begin{examplebox}[Practice question]
\vspace{80mm}
\end{examplebox}


\section*{Introduction to Data Science}

\begin{defbox}
\textbf{1. Classification Problems:} Assign labels 
\begin{itemize}
\item Binary classification
\vspace{10mm}
\item Multiclass classification
\vspace{10mm}
\end{itemize}
\textbf{2. Function Approximation Problems:} Find mathematical relationships between input variables and output responses.
\end{defbox}

\begin{examplebox}[Practice question]
\vspace{70mm}
\end{examplebox}


\subsection*{Why are there so many models?}
\vspace{50mm}

\subsection*{Data Analysis Problem-Solving Framework}

\begin{defbox}[Five-Step Problem-Solving Framework]
\begin{enumerate}
    \item \textbf{Problem Definition}
    \item \textbf{Problem Characterization}
    \item \textbf{Solution Conceptualization}
    \item \textbf{Method Identification}
    \item \textbf{Implementation \& Assessment}
\end{enumerate}
\end{defbox}


\subsection*{Data Imputation}

\vspace{50mm}


\textbf{Matrix Methods for Data Imputation}

\begin{examplebox}[Practice question]
\vspace{240mm}
\end{examplebox}

\begin{examplebox}[Practice question]
\vspace{240mm}
\end{examplebox}


\end{document}